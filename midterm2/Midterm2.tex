\documentclass[]{article}
\usepackage{lmodern}
\usepackage{amssymb,amsmath}
\usepackage{ifxetex,ifluatex}
\usepackage{fixltx2e} % provides \textsubscript
\ifnum 0\ifxetex 1\fi\ifluatex 1\fi=0 % if pdftex
  \usepackage[T1]{fontenc}
  \usepackage[utf8]{inputenc}
\else % if luatex or xelatex
  \ifxetex
    \usepackage{mathspec}
  \else
    \usepackage{fontspec}
  \fi
  \defaultfontfeatures{Ligatures=TeX,Scale=MatchLowercase}
\fi
% use upquote if available, for straight quotes in verbatim environments
\IfFileExists{upquote.sty}{\usepackage{upquote}}{}
% use microtype if available
\IfFileExists{microtype.sty}{%
\usepackage[]{microtype}
\UseMicrotypeSet[protrusion]{basicmath} % disable protrusion for tt fonts
}{}
\PassOptionsToPackage{hyphens}{url} % url is loaded by hyperref
\usepackage[unicode=true]{hyperref}
\hypersetup{
            pdftitle={Midterm2 Stat 151A, Spring 2020},
            pdfauthor={Kanak Garg},
            pdfborder={0 0 0},
            breaklinks=true}
\urlstyle{same}  % don't use monospace font for urls
\usepackage[margin=1in]{geometry}
\usepackage{color}
\usepackage{fancyvrb}
\newcommand{\VerbBar}{|}
\newcommand{\VERB}{\Verb[commandchars=\\\{\}]}
\DefineVerbatimEnvironment{Highlighting}{Verbatim}{commandchars=\\\{\}}
% Add ',fontsize=\small' for more characters per line
\usepackage{framed}
\definecolor{shadecolor}{RGB}{248,248,248}
\newenvironment{Shaded}{\begin{snugshade}}{\end{snugshade}}
\newcommand{\KeywordTok}[1]{\textcolor[rgb]{0.13,0.29,0.53}{\textbf{#1}}}
\newcommand{\DataTypeTok}[1]{\textcolor[rgb]{0.13,0.29,0.53}{#1}}
\newcommand{\DecValTok}[1]{\textcolor[rgb]{0.00,0.00,0.81}{#1}}
\newcommand{\BaseNTok}[1]{\textcolor[rgb]{0.00,0.00,0.81}{#1}}
\newcommand{\FloatTok}[1]{\textcolor[rgb]{0.00,0.00,0.81}{#1}}
\newcommand{\ConstantTok}[1]{\textcolor[rgb]{0.00,0.00,0.00}{#1}}
\newcommand{\CharTok}[1]{\textcolor[rgb]{0.31,0.60,0.02}{#1}}
\newcommand{\SpecialCharTok}[1]{\textcolor[rgb]{0.00,0.00,0.00}{#1}}
\newcommand{\StringTok}[1]{\textcolor[rgb]{0.31,0.60,0.02}{#1}}
\newcommand{\VerbatimStringTok}[1]{\textcolor[rgb]{0.31,0.60,0.02}{#1}}
\newcommand{\SpecialStringTok}[1]{\textcolor[rgb]{0.31,0.60,0.02}{#1}}
\newcommand{\ImportTok}[1]{#1}
\newcommand{\CommentTok}[1]{\textcolor[rgb]{0.56,0.35,0.01}{\textit{#1}}}
\newcommand{\DocumentationTok}[1]{\textcolor[rgb]{0.56,0.35,0.01}{\textbf{\textit{#1}}}}
\newcommand{\AnnotationTok}[1]{\textcolor[rgb]{0.56,0.35,0.01}{\textbf{\textit{#1}}}}
\newcommand{\CommentVarTok}[1]{\textcolor[rgb]{0.56,0.35,0.01}{\textbf{\textit{#1}}}}
\newcommand{\OtherTok}[1]{\textcolor[rgb]{0.56,0.35,0.01}{#1}}
\newcommand{\FunctionTok}[1]{\textcolor[rgb]{0.00,0.00,0.00}{#1}}
\newcommand{\VariableTok}[1]{\textcolor[rgb]{0.00,0.00,0.00}{#1}}
\newcommand{\ControlFlowTok}[1]{\textcolor[rgb]{0.13,0.29,0.53}{\textbf{#1}}}
\newcommand{\OperatorTok}[1]{\textcolor[rgb]{0.81,0.36,0.00}{\textbf{#1}}}
\newcommand{\BuiltInTok}[1]{#1}
\newcommand{\ExtensionTok}[1]{#1}
\newcommand{\PreprocessorTok}[1]{\textcolor[rgb]{0.56,0.35,0.01}{\textit{#1}}}
\newcommand{\AttributeTok}[1]{\textcolor[rgb]{0.77,0.63,0.00}{#1}}
\newcommand{\RegionMarkerTok}[1]{#1}
\newcommand{\InformationTok}[1]{\textcolor[rgb]{0.56,0.35,0.01}{\textbf{\textit{#1}}}}
\newcommand{\WarningTok}[1]{\textcolor[rgb]{0.56,0.35,0.01}{\textbf{\textit{#1}}}}
\newcommand{\AlertTok}[1]{\textcolor[rgb]{0.94,0.16,0.16}{#1}}
\newcommand{\ErrorTok}[1]{\textcolor[rgb]{0.64,0.00,0.00}{\textbf{#1}}}
\newcommand{\NormalTok}[1]{#1}
\usepackage{graphicx,grffile}
\makeatletter
\def\maxwidth{\ifdim\Gin@nat@width>\linewidth\linewidth\else\Gin@nat@width\fi}
\def\maxheight{\ifdim\Gin@nat@height>\textheight\textheight\else\Gin@nat@height\fi}
\makeatother
% Scale images if necessary, so that they will not overflow the page
% margins by default, and it is still possible to overwrite the defaults
% using explicit options in \includegraphics[width, height, ...]{}
\setkeys{Gin}{width=\maxwidth,height=\maxheight,keepaspectratio}
\IfFileExists{parskip.sty}{%
\usepackage{parskip}
}{% else
\setlength{\parindent}{0pt}
\setlength{\parskip}{6pt plus 2pt minus 1pt}
}
\setlength{\emergencystretch}{3em}  % prevent overfull lines
\providecommand{\tightlist}{%
  \setlength{\itemsep}{0pt}\setlength{\parskip}{0pt}}
\setcounter{secnumdepth}{0}
% Redefines (sub)paragraphs to behave more like sections
\ifx\paragraph\undefined\else
\let\oldparagraph\paragraph
\renewcommand{\paragraph}[1]{\oldparagraph{#1}\mbox{}}
\fi
\ifx\subparagraph\undefined\else
\let\oldsubparagraph\subparagraph
\renewcommand{\subparagraph}[1]{\oldsubparagraph{#1}\mbox{}}
\fi

% set default figure placement to htbp
\makeatletter
\def\fps@figure{htbp}
\makeatother


\title{Midterm2 Stat 151A, Spring 2020}
\author{Kanak Garg}
\date{April 12, 2020}

\begin{document}
\maketitle

\section{Introduction}\label{introduction}

Capital Bikeshare System is a bike rental service in our countries
capital, at Washington D.C. Luckily, Capital Bikeshare System has been
collecting information over the course of the last two years regarding
their rental service; gathering information such as the date, the
season, the weather conditions and more. In this report, we will touch
on using this information to see if we can accurately predict bike
rentals in a given hour using this information.

\section{Data Description}\label{data-description}

Each row of the data structure correlates to the amount of bike rentals
for a specific hour over the course of two years. The row comprises of
different variables corresponding to time and weather. Time is divided
up by the different specifications for the time of the purchase -- for
example the month, year, season, year. The data also tracks the weather,
as different weather conditions that may affect bike rental. Many of the
columns in the Bike Rental Data are markers, such as season or year,
that are incorrectly configured as integers. In order for the data to be
usable, these columns must be converted from integer values to the type
factor. The DTEDAY column must be converted to a Date format, as it is a
Date.

\begin{verbatim}
##   instant     dteday season yr mnth hr holiday weekday workingday weathersit
## 1       1 2011-01-01      1  0    1  0       0       6          0          1
## 2       2 2011-01-01      1  0    1  1       0       6          0          1
## 3       3 2011-01-01      1  0    1  2       0       6          0          1
## 4       4 2011-01-01      1  0    1  3       0       6          0          1
## 5       5 2011-01-01      1  0    1  4       0       6          0          1
## 6       6 2011-01-01      1  0    1  5       0       6          0          2
##   temp  atemp  hum windspeed casual registered cnt
## 1 0.24 0.2879 0.81    0.0000      3         13  16
## 2 0.22 0.2727 0.80    0.0000      8         32  40
## 3 0.22 0.2727 0.80    0.0000      5         27  32
## 4 0.24 0.2879 0.75    0.0000      3         10  13
## 5 0.24 0.2879 0.75    0.0000      0          1   1
## 6 0.24 0.2576 0.75    0.0896      0          1   1
\end{verbatim}

\section{Exploratory Data Analysis}\label{exploratory-data-analysis}

Of the 17 data variables that are given to us in the Bike Rental Data,
we can logically eliminate four of the columns in order to end up with a
remaining working data set. The columns INSTANT is a unique number that
corresponds to a row of the dataset, which is useful in data management,
however unneccessary for linear modeling. The column dteday is very
useful as well as it tells us the date for every purchase, however the
rest of our time indicator variables are sufficient and makes this
column unneccessary. The last two columns requiring analysis are columns
registered and casual, which correspond to the type of customer making a
rental. Since CNT = Casual + Registered, a linear model of CNT would be
equivalent to making a linear model of Casual and linear model of
Registered and adding it together. Thus we can further clean data by
removing these two columns.

Further inspecting the dataset, we must divide the variables into
categorical and continous variables. We must inspect both to determine
the validity of these input variables through univariate and bivariate
variable analysis.

\subsubsection{Univariate Variable
Analysis}\label{univariate-variable-analysis}

There were some important charcteristics about the data that stood out
during univariate data analysis. The rest of the continuous data types
seemed to have no skew, thus are fairly normal. Another interesting
characteristic of the data is the variable HOLIDAY, which has over 16000
datapoints for non holidayss, and 500 datapoints for holidays. This is
not ideal as we would like to have more HOLIDAY data values to make a
more sound determination. The rest of the categorical variables have an
even distribution.

\begin{verbatim}
##     instant          dteday               season            yr        
##  Min.   :    1   Min.   :2011-01-01   Min.   :1.000   Min.   :0.0000  
##  1st Qu.: 4346   1st Qu.:2011-07-04   1st Qu.:2.000   1st Qu.:0.0000  
##  Median : 8690   Median :2012-01-02   Median :3.000   Median :1.0000  
##  Mean   : 8690   Mean   :2012-01-02   Mean   :2.502   Mean   :0.5026  
##  3rd Qu.:13034   3rd Qu.:2012-07-02   3rd Qu.:3.000   3rd Qu.:1.0000  
##  Max.   :17379   Max.   :2012-12-31   Max.   :4.000   Max.   :1.0000  
##                                                                       
##       mnth            hr        holiday   weekday  workingday weathersit
##  5      :1488   16     :  730   0:16879   0:2502   0: 5514    1:11413   
##  7      :1488   17     :  730   1:  500   1:2479   1:11865    2: 4544   
##  12     :1483   13     :  729             2:2453              3: 1419   
##  8      :1475   14     :  729             3:2475              4:    3   
##  3      :1473   15     :  729             4:2471                        
##  10     :1451   12     :  728             5:2487                        
##  (Other):8521   (Other):13004             6:2512                        
##       temp           atemp             hum           windspeed     
##  Min.   :0.020   Min.   :0.0000   Min.   :0.0000   Min.   :0.0000  
##  1st Qu.:0.340   1st Qu.:0.3333   1st Qu.:0.4800   1st Qu.:0.1045  
##  Median :0.500   Median :0.4848   Median :0.6300   Median :0.1940  
##  Mean   :0.497   Mean   :0.4758   Mean   :0.6272   Mean   :0.1901  
##  3rd Qu.:0.660   3rd Qu.:0.6212   3rd Qu.:0.7800   3rd Qu.:0.2537  
##  Max.   :1.000   Max.   :1.0000   Max.   :1.0000   Max.   :0.8507  
##                                                                    
##      casual         registered         cnt       
##  Min.   :  0.00   Min.   :  0.0   Min.   :  1.0  
##  1st Qu.:  4.00   1st Qu.: 34.0   1st Qu.: 40.0  
##  Median : 17.00   Median :115.0   Median :142.0  
##  Mean   : 35.68   Mean   :153.8   Mean   :189.5  
##  3rd Qu.: 48.00   3rd Qu.:220.0   3rd Qu.:281.0  
##  Max.   :367.00   Max.   :886.0   Max.   :977.0  
## 
\end{verbatim}

\subsubsection{Bivariate Variable
Analysis}\label{bivariate-variable-analysis}

Onto a bivariate analysis of the variables. We see a paired similarity
between the variables SEASON and MONTH. Graphing the boxplot of both,
shows that month may be a better categorical variable to use as there
seems to be differences between months in a season, however not in
seasons in months (only a few months have multiple seasons per month,
most only have one). Using the pairs graph we analysze all the
continuous variables below. As in Figure 1, there seems to be
correlation between TEMP and ATEMP. Since ATEMP has a smaller
z-statistic, we decided to use ATEMP.

Analyzing the paired relationship between WEEKDAY,WORKINGDAY and
HOLIDAY, I believe we condense the variables to only one or two.
WEEKDAYs is the broadest variable, while WORKINGDAY and HOLIDAY gives us
more information. The difference in average rentals for WORKINGDAY and
HOLIDAY seem to be significant enough to keep in the data. Furthermore
from the pair analysis, we should remove WEEKDAYS, however we dont have
enough information from the data and should wait till model selection to
decide.

\begin{figure}
\centering
\includegraphics{Midterm2_files/figure-latex/unnamed-chunk-2-1.pdf}
\caption{\label{fig:vars} Figure 1: Paired plot of all the the weather
data, which helps determine which of the weather data can be used to
interpret CNT.}
\end{figure}

\section{Model Selection}\label{model-selection}

Now that we have narrowed down the parameters to a usable set of
variables, we want to tranform our data and determine which of these
variables are necessary in order to determine an accurate model.

\subsubsection{Transforming CNT}\label{transforming-cnt}

After removing the variables we talked about above, we must look at CNT
relationship with the input variables. A plot of the fully fitted linear
model is below in Figure 2. Looking at the outputted graphs, we see a
slight curve in the residuals as well as a a graph that lacks fit in the
initial and end of the q-q plot. The scale location model seems to
increase toward the end of the graph, hinting at a lack of
homoscedascity. The last residuals vs leverage graph shows us 3 data
points that are high leverage and could thus be influential points.

\begin{figure}
\centering
\includegraphics{Midterm2_files/figure-latex/unnamed-chunk-4-1.pdf}
\caption{\label{fig:lm} Figure 2: Fully fitted data model of Count data
using all explanatory variables provided in the dataset.}
\end{figure}

The best way to deal with this is by taking a log of the CNT data.

\subsubsection{Influential Points}\label{influential-points}

In the linear model of Figure 2, we also see three important high
leverage points at (586,9124,8855). These points are not necessarily a
bad sign - however in order to determine a correct model, we remove
these points in order to prevent any change they may have on the data.
However, as they properly align with the rest of the data, as seen in
the Residual vs Leverage graph, there is a low chance that they will
severly impact our linear regression.

\begin{figure}
\centering
\includegraphics{Midterm2_files/figure-latex/unnamed-chunk-5-1.pdf}
\caption{\label{fig:lmtrans} Figure 3: Graphing the linear model after
making transformations to output variables and removing leverage
points.}
\end{figure}

The resulting linear model, Figure 3, is a lot better as the Residual vs
Fitted graph are more centered around the mean and the Residuals vs
Leverage is more linear. Some points of concern include a dip in
Scale-Location graph toward later values, and a lag in initial values in
the Q-Q Plot. This may indicate slight heteroscedascity, where values
have less variance almost halfway through the dataset. Also our data
seems to be less normal in the initial half of the dataset. Since the
data is divided into two, as seen in the plot of CNT, as well as the Q-Q
plot and Scale-Location, there may be some interaction present with the
variable YR as it is the only variable that divides the dataset in half.

\subsubsection{Forward Selection}\label{forward-selection}

We have many options for determining such model -- including forward
selection, backward selection, stepwise selection or even Lasso and
Ridge regression. In this model we determine to use a forward selection
model as it seems the most appropiate and aggregate tests over many
different factors. Though backward selection would also be appropiate,
attempting to run Lasso and Ridge did not seem to be successful due to
the presence of many categorical variables in the data.

In forward selection, the best way to determine if a model is appropiate
is to start with an empty model and filter in variables based on which
seems to be the most appropiate to the model. We use many different
criteria -- p-values, AIC, BIC, Mallows C'p and Adjusted R\^{}2. The way
we perform forward selection is by creating a model and adding the best
variable one at a time as long as the criteria is met. The set of
variables we decided to use are the basic variables in the data as well
as year's interaction with the other time categorical variables.

Based on every criteria that we perform forward selection, we end up
with the same model. The best model for predicting the data seems to
contain all input variables as well as time variables and their
interaction with YEAR. We end up with 3 main models --

\textbf{Forward selection (p-val)} \{``hr'' ``yr'' ``mnth''
``weathersit'' ``atemp'' ``yr*mnth" ``weekday'' ``hum'' ``holiday''\}

\textbf{Forward selection (adjusted R\^{}2) } \{``windspeed''
``holiday'' ``hum'' ``yr*weekday" ``atemp'' ``weathersit'' ``yr*mnth"
``yr*hr" \}

\textbf{Forward selection (AIC,BIC,Mallows Cp)} \{``yr*hr" ``yr*mnth"
``weathersit'' ``atemp'' ``weekday'' ``hum'' ``holiday'' ``windspeed''\}

\subsubsection{Selecting Model}\label{selecting-model}

Of the three models, we prefer the model that can predict the best. Thus
we use cross validation in order to determine which model has the
highest probability of predicting future values. Of the three, the
forward selection with p-val proved to have the highest probability,
even though by a slight value. Thus the model we chose is Model 1. The
probabibilities are printed below in order as its listen above.

\begin{verbatim}
## [1] 0.6294442 0.6284303 0.6282918
\end{verbatim}

\section{Interpretation}\label{interpretation}

Our chosen model is -

\(log(CNT) = A + b_{hr} hr + b_{yr} yr + b_{mnth} mnth + br_{yxm} yr*mnth + br_{wd} weekday + br_{holiday} holiday + b_{ws} weathersit + b_{atemp} atemp + br_{hum} hum\)

Thus the total count of rentals procured in an hour is dependent on
which hour of the day it is, the month, year, weekday, weather situation
outside, how it feels outside, the humidity, whether or not today is a
holiday and a interaction between the year and month element.

There are two types of variables used in our model - continous and
categorical variables. The continous variables in the model are ATEMP
and HUM.

\begin{itemize}
\item
  \textbf{ATEMP} is the temperature that it feels like outside. This
  means that if the ATEMP values were to max out at 50, to reach a value
  of 1, then our model would have a postive increase of 1.433 for the
  log(CNT). As the temperature increases, the model predicts more
  rentals.
\item
  \textbf{HUM} is the humidity normalized. If the humidity were to max
  out at 100, to reach a value of 1, our model would decrease by a value
  0.26124 for the log(CNT). AS the humidity increase, the model predicts
  less rentals.
\end{itemize}

The categorical variables are HR, YR, MNTH, WEEKDAY, HOLIDAY, and
WEATHERSIT.

\begin{itemize}
\item
  \textbf{HR} is the hour at which the rental was made. The natural
  value for this variable is set at hour 0 and all our data can be an
  interpretation relative to midnight of the day. We see an initial
  decrease in rentals from 12 - 5 AM, but then a sudden increase in
  rentals from 6 AM to 5 PM. 6 PM and onward the rentals decrease but
  end off on an overall increase from the day start.
\item
  \textbf{MNTH/YR} is the month and year of whihc the rental was made.
  The natural setting for this variable is year 0, January. As we move
  to the year 1, There is an on average increase of 0.78 in log(CNT). As
  we move from January to the other months, the amount of rentals tend
  to increase and follow a bimodal trend with a max on May and October.
  Thus as the years progress and was we travel through the year, the
  amount of rentals increase. We also included an interaction between
  MNNTH and YEAR, as we postulated a difference in rentals between the
  years. Overall, the affect of time on rentals comes out to (0.78 - the
  coefficient of yr:mnthx + mnthx) for months in year1, an overall
  increase from the year before.
\item
  \textbf{WEEKDAY} is the weekday from our data set. The natural setting
  is sunday, and the rest of the data is relative to sunday on the week.
  We see the most rentals on Friday and Saturday and a mediocre amount
  of rentals the rest of the week.
\item
  \textbf{HOLIDAY} is an indicator variable determining if the day is a
  holiday or not. Though there were not many holiday variables, the
  difference between the rentals on non holiday days and holiday days
  were significant enough to include in our model. The average amount of
  rentals decrease by 0.1874 when there is a holiday.
\item
  \textbf{WEATHERSIT} is an indicator of the weather during the day. The
  best weather for rentals is waether type 1 -- clear out with few
  clouds. As the weather progresses through the types, becoming more
  severe, the amount of bike rentals go down. Something interesting to
  notice is the lack of a weathersit4 predictor, whihc may be due to the
  lack of enough samples for this data point.
\end{itemize}

The last point to interpret is our intercept, which lets us know default
values. In the case that we were in year 0, midnight on Sunday in
January which is not a holiday but has good weather, with a humidity of
0 and felt like 0 degrees, we predict to have a 2.4279 log(cnt) of sales
in that day.

\section{Conclusion}\label{conclusion}

Overall I beleive this model makes sense, as the business continues to
grow through time the amount of rentals should increase. As well as
important factors about the day, whether people tend to work on that
day, whether it is a holiday, if the time of the year is good or not,
all could be factors for bike rentals for customer. Humidity and the
temperature it feels like outside are also important factors for
customer's chosing to do a outdoors activity. Thus these variables can
appropiately and logically be placed in the model. In further bikeshare
use, businesses would benefit from marketing on days that are more
favorable to them, as they can predict more bike use on that day.

This model also seems to be the best fit as it has a some of the top
R\^{}2, high AIC, BIC, and Mallow's Cp compared to other models and the
model we used in the beginning. Thus through all interpretation, this
model seems to be the best fit.

\(log(CNT) = A + b_{hr} hr + b_{yr} yr + b_{mnth} mnth + br_{yxm} yr*mnth + br_{wd} weekday + br_{holiday} holiday + b_{ws} weathersit + b_{atemp} atemp + br_{hum} hum\)

\newpage

\section{APPENDIX}\label{appendix}

\begin{Shaded}
\begin{Highlighting}[]
\KeywordTok{hist}\NormalTok{((bike}\OperatorTok{$}\NormalTok{cnt)}\OperatorTok{^}\NormalTok{(}\DecValTok{1}\OperatorTok{/}\DecValTok{2}\NormalTok{))}
\KeywordTok{abline}\NormalTok{(}\DataTypeTok{v =} \KeywordTok{mean}\NormalTok{((bike}\OperatorTok{$}\NormalTok{cnt)}\OperatorTok{^}\NormalTok{(}\DecValTok{1}\OperatorTok{/}\DecValTok{2}\NormalTok{)), }\DataTypeTok{col =} \StringTok{"blue"}\NormalTok{, }\DataTypeTok{lwd =} \DecValTok{2}\NormalTok{)}
\KeywordTok{abline}\NormalTok{(}\DataTypeTok{v =} \KeywordTok{median}\NormalTok{((bike}\OperatorTok{$}\NormalTok{cnt)}\OperatorTok{^}\NormalTok{(}\DecValTok{1}\OperatorTok{/}\DecValTok{2}\NormalTok{)), }\DataTypeTok{col =} \StringTok{"red"}\NormalTok{, }\DataTypeTok{lwd =} \DecValTok{2}\NormalTok{)}
\end{Highlighting}
\end{Shaded}

\includegraphics{Midterm2_files/figure-latex/unnamed-chunk-12-1.pdf}

\begin{Shaded}
\begin{Highlighting}[]
\CommentTok{#determining workday vs weekday vs holiday}

\KeywordTok{mean}\NormalTok{(rntDAT[}\KeywordTok{which}\NormalTok{(rntDAT}\OperatorTok{$}\NormalTok{weekday }\OperatorTok\StringTok{ }\KeywordTok{c}\NormalTok{(}\DecValTok{0}\NormalTok{,}\DecValTok{6}\NormalTok{)),]}\OperatorTok{$}\NormalTok{cnt)}
\end{Highlighting}
\end{Shaded}

\begin{verbatim}
## [1] 4.554327
\end{verbatim}

\begin{Shaded}
\begin{Highlighting}[]
\KeywordTok{mean}\NormalTok{(rntDAT[}\KeywordTok{which}\NormalTok{(rntDAT}\OperatorTok{$}\NormalTok{weekday }\OperatorTok\StringTok{ }\KeywordTok{c}\NormalTok{(}\DecValTok{1}\NormalTok{,}\DecValTok{2}\NormalTok{,}\DecValTok{3}\NormalTok{,}\DecValTok{4}\NormalTok{,}\DecValTok{5}\NormalTok{)),]}\OperatorTok{$}\NormalTok{cnt)}
\end{Highlighting}
\end{Shaded}

\begin{verbatim}
## [1] 4.528829
\end{verbatim}

\begin{Shaded}
\begin{Highlighting}[]
\KeywordTok{summary}\NormalTok{(rntDAT}\OperatorTok{$}\NormalTok{cnt)}
\end{Highlighting}
\end{Shaded}

\begin{verbatim}
##    Min. 1st Qu.  Median    Mean 3rd Qu.    Max. 
##   0.000   3.689   4.956   4.536   5.638   6.884
\end{verbatim}

\begin{Shaded}
\begin{Highlighting}[]
\KeywordTok{sd}\NormalTok{(rntDAT}\OperatorTok{$}\NormalTok{cnt)}
\end{Highlighting}
\end{Shaded}

\begin{verbatim}
## [1] 1.486204
\end{verbatim}

\begin{Shaded}
\begin{Highlighting}[]
\CommentTok{#mnth vs season}
\KeywordTok{boxplot}\NormalTok{(cnt }\OperatorTok{~}\StringTok{  }\NormalTok{season }\OperatorTok{+}\StringTok{ }\NormalTok{mnth , }\DataTypeTok{data =}\NormalTok{ bike)}
\end{Highlighting}
\end{Shaded}

\includegraphics{Midterm2_files/figure-latex/unnamed-chunk-12-2.pdf}

\begin{Shaded}
\begin{Highlighting}[]
\CommentTok{#temp vs atemp}
\KeywordTok{gpairs}\NormalTok{(bike[,}\KeywordTok{c}\NormalTok{(}\StringTok{"temp"}\NormalTok{,}\StringTok{"atemp"}\NormalTok{, }\StringTok{"hum"}\NormalTok{, }\StringTok{"windspeed"}\NormalTok{, }\StringTok{"weathersit"}\NormalTok{)])}
\end{Highlighting}
\end{Shaded}

\includegraphics{Midterm2_files/figure-latex/unnamed-chunk-12-3.pdf}

\end{document}
